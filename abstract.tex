\documentclass[11pt]{article}
\usepackage{fontspec}
\setmainfont{Helvetica}
\usepackage[margin=0.5in]{geometry}
\usepackage[singlespacing]{setspace}
\usepackage[small, compact]{titlesec}  % http://ctan.mirrors.hoobly.com/macros/latex/contrib/titlesec/titlesec.pdf
\titleformat*{\subsubsection}{\slshape}
\usepackage{paralist}
\usepackage[english]{babel}
\pagenumbering{gobble}
\usepackage{hyperref}
\hypersetup{colorlinks=true, urlcolor=blue, linkcolor=black, citecolor=black}

\begin{document}
\sloppy
\begin{center}
\large{\textbf{
    Improving Machine Learning Models of the Human Gut Microbiome
}}

\vspace{11pt}

\small{
    Kelly L. Sovacool
}
\end{center}

Changes in the taxonomic composition and metabolic activity of human microbiomes
have been observed in several diseases including colorectal cancer (CRC) and
\textit{Clostridioides difficile} infection (CDI).
Taxonomic composition is commonly defined by amplicon sequencing the 16S rRNA
gene and clustering sequences into Operational Taxonomic Units (OTUs).
The OTU abundances can then be used to train supervised machine learning 
models for tasks such as classifying samples as CRC or normal, or predicting the
severity of CDI outcomes.
Such models have the potential to improve the early detection of CRC, inform
clinicians on which CDI patients may be most at risk of experiencing a severe
case, or more generally contribute to our understanding of how the gut
microbiome changes during disease states.
However, there are a number of challenges to realizing the full 
potential of machine learning models of the human gut microbiome.
First, current \textit{de novo} OTU clustering methods produce high quality
OTUs, but OTU assignments may change when new data are added.
When deploying machine learning models trained on OTUs, external validation sets
need to have the same OTUs as the data used for initial model training, which is
not currently possible without reference-based methods that produce lower
quality OTUs.
Second, efforts to find consistent changes in taxonomic composition of
microbiomes between normal and dysbiotic states have found mixed success, in
part because interpersonal variability in taxonomic composition sometimes
exceeds the variability between disease states.
Variability of microbiome composition between individuals with the same disease
status may be explained by functional redundancy, where different microbial
species carry out the same functions and thus can replace each other with little
effect on the overall function of the community.
This proposal aims to 
1) develop a new OTU clustering method to enable new data to be fit to existing
OTUs, 
2) train OTU-based machine learning models to predict CDI severity on a large
dataset and apply the new clustering method for validation with an external
dataset, and 
3) incorporate functional composition from metagenomic data to improve the
performance of machine learning models for classifying CRC cases.

To fill the need for a stable clustering method that produces high quality OTUs,
we are developing OptiFit, a new algorithm that assigns sequences to
existing \textit{de novo} OTUs created by OptiClust.
OptiFit is being refined and tested to determine its performance compared to
OptiClust using publicly available datasets from human gut, mouse gut, marine,
and soil microbiomes.
The OTU quality and execution speed will be compared between OptiFit and
OptiClust for all four datasets across 100 different random seeds.
We hypothesize that OptiFit clusters reference-based OTUs at nearly the same 
quality as OptiClust and with faster execution speeds, allowing researchers to 
fit new data to existing OTUs for model validation and deployment.

Previous studies have built OTU-based machine learning models to distinguish CDI
cases from controls in order to demonstrate the importance of the gut microbiome
in CDI and to identify the most important microbial features contributing to
model performance.
We now have access to a large dataset of amplicon sequences and
clinical features from about 4,000 stool samples of CDI cases, diarrheal
controls, and non-diarrheal controls (R21 cohort).
OTU-based machine learning models will be built to not only distinguish CDI
cases but also to predict the severity of infection in CDI cases based on
clinical lab results.
The OptiFit algorithm will then be used to cluster an external dataset (ERIN
cohort) to the OTUs from the R21 cohort, allowing the external dataset to be
used as a validation set.
A model trained on such a large dataset and validated with external data could
be used to help clinicians identify cases that may become severe in order to
inform treatment strategies.

Previous studies have built OTU-based machine learning models to classify stool
samples as normal or cancerous to serve as a less invasive diagnostic tool for
CRC than colonoscopy, but have only achieved modest performance.
Incorporating the known functional potential of the microbiome from metagenomic
data may help account for functional redundancy and improve the performance of
OTU-based models in classifying CRC, predicting CDI severity, or other 
microbiome modeling problems.
We have whole metagenomes from colorectal carcinoma, adenoma, and normal stool
samples.
We will first identify known gene functions present in the samples and
characterize the functional redundancy within and between disease states, then
train machine learning models with known gene functions, OTUs, or both as
features to classify samples into disease states.
We hypothesize that models trained with known gene functions as features 
would outperform models built on OTUs alone.

\end{document}
